%!TeX root=../tese.tex
\chapter{Resultados clássicos}
\label{cap:classicos}


Seja $G$ um grafo. Defininimos $D(G)$ como o menor tamanho de um $F \subseteq E(G)$ tal que $G-F$ é bipartido.

\begin{theorem}[Mantel]
  Seja $G$ um grafo livre de triângulos com $n$ vértices.
  Então $e(G) \leq \left\lfloor\frac{n^2}{4}\right\rfloor$.
  Além disso, se vale a igualdade então $G$ é bipartido completo.
\end{theorem}

\begin{theorem}[Estabilidade] \label{thm:estabilidade}
  Seja $m \geq 0$ um inteiro e seja $G$ um grafo livre de triângulos com $n$ vértices e $\frac{n^2}{4}-m$ arestas.
  Então $D(G) \leq m$.
\end{theorem}

\begin{conjecture}[Erd\H os] \label{conj:make-bipartite}
  Seja $G$ um grafo livre de triângulos com $n$ vértices.
  Então $G$ pode ser tornado bipartido pela remoção de no máximo $\frac{n^2}{25}$ arestas,
  i.e. \[D(G) \leq \frac{n^2}{25}.\]
\end{conjecture}

Uma Conjectura relacionada:

\begin{conjecture} \label{conj:metadinha}
  Seja $G$ um grafo livre de triângulos com $n$ vértices.
  Então existe $X \subseteq V(G)$ com $X = \left\lfloor \frac{n}{2} \right\rfloor$ tal que $e(G[X]) \leq \frac{n^2}{50}$.
\end{conjecture}

Observe que o Teorema \ref{thm:estabilidade} prova a Conjetura para grafos suficientemente densos (com pelo menos $\frac{n^2}{4} - \frac{n^2}{25}$ arestas).

\begin{definition}
  Sejam $G$ um grafo e $H$ um blow-up de $G$, com $\phi \colon V(H) \to V(G)$ sendo um homomorfismo que define esse blow-up.
  Dizemos que um $S \subseteq E(H)$ é \emph{canônico com relação a $\phi$} se para quaisquer $e,f \in E(H)$ com
  $\phi(e)=\phi(f)$ vale que $e \in S \iff f \in S$.
  Em outras palavras, entre cada par de classes de $H$ escolhemos ou todas as arestas entre essas classes ou não escolhemos nenhuma dessas arestas.
  
  Se $\phi$ for claro do contexto, iremos omitir e dizer apenas que o conjunto de arestas do blow-up é simplesmente \emph{canônico}.
\end{definition}

\begin{theorem}[Simetrização]
  Seja $G$ um grafo livre de triângulos e seja $H$ um blow-up de $G$.
  Então existe $F \subseteq E(H)$ canônico com $|F|=D(H)$ e tal que $G-F$ é bipartido.
\end{theorem}

\begin{corollary}
  Seja $H$ um blow-up balanceado de $C_5$ com $n$ vértices.
  Então \[ D(H) = \frac{n^2}{25}. \]
  Em particular, a Conjectura \ref{conj:make-bipartite} (se verdadeira) dá a melhor constante possível.
\end{corollary}

\begin{theorem}[EFPS] \label{thm:EFPS-bounds}
  Seja $G$ um grafo livre de triângulo com $n$ vértices e $m$ arestas.
  Então \[ D(G) \leq \left\{m-\frac{m^2}{4n}, \frac{m}{2} - \frac{2m(2m^2-n^3)}{n^2(n^2-2m)}\right\}. \]
\end{theorem}

\begin{corollary} \label{cor:n2/5}
  Para todo $n$ inteiro positivo, a Conjectura \ref{conj:make-bipartite}
  é verdadeira para grafos com $n$ vértices e pelo menos $\frac{n^2}{5}$ arestas.
\end{corollary}

\begin{theorem}[Erd\H os - Gy\H ori - Simonovits]
  Seja $G$ um grafo livre de triângulos com $n$ vértices e pelo menos $\frac{n^2}{25}$ arestas.
  Então existe um grafo $H$ também com $n$ vértices tal que $H$ é um blow-up de $C_5$ e, além disso,
  $e(G) \leq e(H)$ e $D(G) \leq D(H)$.
\end{theorem}

A prova é algorítmica.
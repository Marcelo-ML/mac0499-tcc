%!TeX root=../tese.tex
\section{Considerações sobre software e questões numéricas}\label{sec:software}

O método de álgebras de flag foi originalmente introduzido por Razborov em um contexto muito mais geral que de grafos, na linguagem da Teoria dos Modelos Finitos.
De fato, não é difícil generalizar a descrição prática desse capítulo para contextos como de hipergrafos, digrafos, grafos com coloração, permutações, entre vários outros modelos.

Apesar disso, ainda não há uma implementação ``padrão'' das álgebras de flag.
O software~\texttt{flagmatic} (\url{https://lidicky.name/flagmatic/}) é possivelmente a implementação mais acessível e tem sido usada com êxito tipo Turán, o software {flagmatic} já foi usado e testado com resultados prolíficos em diversos problemas de densidade em grafos e hipergrafos.
Outras implementações podem ser encontradas facilmente e tem sido usados com êxito em problemas de combinatória extremal~(ver~\cite{brosch2022symmetry,brosch2025lower,lidicky2021semidefinite}).

% https://github.com/DanielBrosch/SDPSymmetryReduction.jl (Julia)
% flag-algebra crate (Rust)
% permpack (sage)
% flag-ocaml (ocaml)

%Por exemplo, é possível verificar o resultado de~\cite{grzesik2012pentagon} em poucos segundos com uma instalação local do \texttt{flagmatic}. 

Nesse trabalho, optamos por utilizar o pacote~\texttt{flag-algebra-program-package} (desenvolvido por Leonardo Nagami Coregliano e disponível em~\url{https://github.com/robertoparente/flag-algebra-program-package}) para elaborar os programas, que é desenvolvido inteiramente em \texttt{C++} e foi usado para obter os resultados de~\cite{coregliano2015maximum}.
A implementação está disponível em~\url{https://github.com/Marcelo-ML/flag-algebras}.

Escolhemos esse pacote como base porque ele oferecia um nível suficiente de abstração e otimização para certas propriedades como normalização e produto de densidades, além de a documentação e linguagem serem mais acessíveis nas fases iniciais desse trabalho.
Usamos o \texttt{csdp} para resolver os programas semidefinidos vindos da formulação em álgebras de flag.
No futuro, consideraremos revisar os métodos utilizados e trabalhar com uma distribuição do~\texttt{flagmatic}.

Por fim, cabe ressaltar que muitas vezes os cálculos não são feitos de forma exata, e isso depende da implementação utilizada.
Há métodos de arredondamento utilizados (implementados pelo~\texttt{flagmatic}), mas no caso da geração de cortes locais, arredondar os resultados em ponto flutuante dos resultados do~\texttt{csdp} incorreria em ainda mais tempo de execução para os programas.
Como nesse trabalho nos preocupamos, principalmente, com o detalhamento da implementação dos métodos de~\cite{baloghclemenlidicky}, não realizamos cálculos de forma exata.

De fato, em~\cite{baloghclemenlidicky} é fornecido um esboço de prova para o regime de densidade $e(G) \geq (0.2-\varepsilon)n^2$ para $\varepsilon \approx 10^{-8}$ (que depende dos erros máximos garantidos pela execução do \texttt{csdp}).
A prova desse resultado utiliza um argumento de remoção de vértices de grau baixo e também o Teorema~\ref{thm:dG-dH-blowup-C5}.
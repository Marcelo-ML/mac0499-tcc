%!TeX root=../tese.tex
\section{Considerações sobre software e questões numéricas}\label{sec:software}

Até a data da entrega desse trabalho, não há um software unificado utilizado para realizar manipulações com álgebras de flag, mas há diversas implementações do método disponíveis.
%(ver~\cite{parczyk2023fully})
%O Sage ainda não possui algo de álgebra de flag, o que é uma pena.
Em problemas do tipo Turán, o software \href{https://lidicky.name/flagmatic/}{flagmatic} já foi usado e testado com resultados prolíficos em diversos problemas de densidade em grafos e hipergrafos.
Por exemplo, é possível verificar o resultado de~\cite{grzesik2012pentagon} em poucos segundos com uma instalação local do \texttt{flagmatic}. 
%Há uma implementação para rust também.

\text{\color{red} [esse parágrafo depende de se eu vou conseguir debugar essa joça até dia 14]}
Nesse trabalho, optamos por utilizar o pacote~\texttt{flag-algebra-program-package} (desenvolvido por Leonardo Nagami Coregliano e disponível em~\href{https://github.com/robertoparente/flag-algebra-program-package}) para elaborar os programas, porque oferecia uma quantidade suficiente de abstração e maleabilidade para implementar o que precisávamos.
Além disso, não fizemos as computações de forma exata.
O \texttt{flagmatic} faz isso, mas aqui por simplicidade, para observar de forma empírica métodos de~\cite{baloghclemenlidicky}, fizemos numérico mesmo.
Ou seja, seria necessário verificar o resutado para $e(G) \geq (0.2-\epsilon)n^2$, onde $\epsilon$ vem da precisão da máquina.
De fato, em~\cite{baloghclemenlidicky} fornecem um esboço de prova para esse regime de densidade, usando o Teorema~\ref{thm:dG-dH-blowup-C5}.
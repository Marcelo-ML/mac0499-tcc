%!TeX root=../tese.tex
%("dica" para o editor de texto: este arquivo é parte de um documento maior)
% para saber mais: https://tex.stackexchange.com/q/78101

% As palavras-chave são obrigatórias, em português e em inglês, e devem ser
% definidas antes do resumo/abstract. Acrescente quantas forem necessárias.
\palavraschave{Grafos, Combinatória, Álgebras de flag}

\keywords{Graphs, Combinatorics, Flag algebras}

% O resumo é obrigatório, em português e inglês. Estes comandos também
% geram automaticamente a referência para o próprio documento, conforme
% as normas sugeridas da USP.
\resumo{
Grafos livres de triângulos são objetos de grande importância na área de Teoria Extremal dos Grafos.
O clássico Teorema de Mantel mostra o limite para o número máximo de arestas que um tal grafo pode ter, e a partir da década de 1960, perguntas mais desafiadoras foram postas relacionadas a esses grafos.
Nesse trabalho, nos ocupamos com o estudo de questões de estabilidade em grafos livres de triângulos, e particularmente com uma conjectura proposta por Erd\H os em 1975 sobre a distância entre grafos livres de triângulos e grafos bipartidos.

Inicialmente, apresentamos resultados parciais para a conjectura utilizando técnicas clássicas em Teoria Extremal dos Grafos.
Em seguida, apresentamos e aplicamos duas técnicas modernas para avançar na conjectura: as álgebras de flag de Razborov, que permitem automatizar certas estratégias de prova que generalizam os métodos clássicos usando cortes locais, e teoremas de homomorfismos em grafos com restrição de grau mínimo, que facilitam o estudo de objetos complexos a partir de uma perspectiva mais simples.
Obtemos avanços parciais na conjectura principal utilizando os dois métodos, que se complementam pela força expressiva e versatilidade computacional das álgebras de flag e pela simplificação estrutural dos teoremas de homomorfismo.
}

\abstract{
Triangle-free graphs are objects of great importance in Extremal Graph Theory.
A classic result by Mantel establishes the limit for the maximum number of edges a triangle-free graph can have. Starting in the 1960s, more challenging questions have been posed about such graphs.
In this work, we investigate stability questions related to triangle-free graphs, particularly a conjecture proposed by Erd\H os in 1975 regarding the distance between triangle-free graphs and bipartite graphs.

Initially, we present partial results to the conjecture using classical techniques in Extremal Graph Theory.
Then, we present and apply two modern techniques to advance the conjecture: Razborov’s flag algebras, which allow for the automatization of certain proof strategies that generalize the classical methods using local cuts, and homomorphism theorems in graphs with minimum degree restrictions, which facilitate the study of complex objects by offering a simpler perspective.
We obtain partial results for the conjecture using both methods, which complement each other through the expressive power and computational versatility of the flag algebras and the structural simplification afforded by the homomorphism theorems.}

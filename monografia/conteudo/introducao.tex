%!TeX root=../tese.tex
\chapter**{Introdução}

Uma das perguntas mais fundamentais em Teoria Extremal dos Grafos é ``qual o maior número possível de arestas em um grafo de $n$ vértices sem cópias de $H$ como subgrafo?''
Os celebrados teoremas de Turán e Erd\H os-Stone, e Erd\H os-Kovari-S\'os dão respostas complementares para essa pergunta quantitativa, e muitas vezes (como no caso dos dois primeiros) fornecem tanto respostas qualitativas.
De fato, se $H$ não é bipartido, o Teorema de Erd\H os-Stone diz que um grafo que atinja a cota superior deve se aproximar de um grafo $(\chi(H)-1)$-partido, onde $\chi(H)$ é o número cromático de $H$.
Para mais detalhes, recomendamos o excelente livro~\cite{livrocombinatoria}.

A resposta (numérica e estrutural) para essa pergunta no caso em que $H$ é um \textit{triângulo} é um dos resultados mais antigos em Teoria Extremal dos Grafos, tendo sido provado por Mantel em 1907: um grafo em $n$ vértices sem triângulos possui no máximo $n^2/4$ arestas, e se a cota é atingida com igualdade, então $H$ é bipartido.
A partir das décadas de 1960 e 1970, com o trabalho de Erd\H os, Simonovits, Andrásfai e outros, a interação entre restrições numéricas e restrições estruturais em grafos introduz perguntas diversas, entre elas o estudo da \emph{estabilidade}.

A estabilidade se refere justamente ao comportamento de grafos que, próximos ao limiar para o qual uma propriedade acontece, se aproximam de um exemplo extremal.
Para o Teorema de Mantel, uma pergunta que se pode fazer é ``quão próximo está um grafo livre de triângulos com muitas arestas de ser bipartido''?
Em 1975, Erd\H os conjecturou uma forte resposta para essa pergunta, que todo grafo livre de triângulos com $n$ vértices de fato pode ser tornado bipartido deletando no máximo $n^2/25$ das suas arestas.

O caso geral dessa conjectura permanece em aberto até a data da conclusão desse trabalho.
Nesse trabalho, realizamos um estudo da conjectura, dos resultados clássicos provados em direção a estabilidade em grafos livres de triângulos e posteriormente apresentamos e empregamos duas ferramentas modernas (álgebras de flag e teoremas de homomorfismos relacionados a condições de grau mínimo) para obter avanços parciais na direção do caso geral da Conjectura.

\section*{Estrutura do trabalho}

Organizamos os capítulos subsequentes como segue.

\begin{itemize}
	\item No Capítulo~\ref{cap:preliminares}, introduzimos alguns conceitos e fixamos a notação que será utilizada ao longo deste trabalho.
	\item No Capítulo~\ref{cap:classicos}, discutimos uma breve história do problema e resultados parciais clássicos.
	\item No Capítulo~\ref{cap:flag-algebras}, apresentamos as álgebras de flag e a técnica de cortes locais para obter resultados numéricos na direção da Conjectura~\ref{conj:make-bipartite}.
	\item No Capítulo~\ref{cap:grau-limitado}, discutimos teoremas de homomorfismo e a Conjectura~\ref{conj:make-bipartite} para grafos de grau mínimo alto.
	\item No Capítulo~\ref{cap:conclusao}, discutimos brevemente os resultados obtidos e direções futuras de pesquisa.
\end{itemize}

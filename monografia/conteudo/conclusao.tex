%!TeX root=../tese.tex
\chapter{Conclusão}

Nesse trabalho, apresentamos a Conjectura~\ref{conj:make-bipartite} de Erd\H os, que completa 50 anos em aberto no ano de 2025.
No Capítulo~\ref{cap:classicos}, estudamos resultados parciais clássicos para a conjectura, utilizando contagens e simetrização que remontam às décadas de 1970 e 1980.
Nos Capítulos~\ref{cap:flag-algebras} e~\ref{cap:grau-limitado}, propusemos estudar a Conjectura~\ref{conj:make-bipartite} utilizando técnicas mais modernas de teoria extremal de grafos: as álgebra de flag de Razborov e os Teoremas de~\cite{jin199510n29} e~\cite{brandt2011vega}.

Os métodos utilizados no Capítulo~\ref{cap:flag-algebras} são mais robustos e o Teorema~\ref{baloghclemenlidicky} e cobrem os resultados do Capítulo~\ref{cap:grau-limitado}, uma vez que grafos com grau mínimo maior que $n/3$ tem mais de $n^2/6$ arestas.

Não se descarta a importância do estudo continuado de propriedades estruturais de blow-ups de Andrásfai para a Conjectura~\ref{conj:make-bipartite}, em particular para compreender a dinâmica do problema em blow-ups de grafos conhecidos ou os exemplos extremais abaixo do limiar de arestas $n^2/5$ (não coberto pelo Teorema~\ref{thm:dG-dH-blowup-C5}).

Por fim, destaca-se mais uma vez a relação entre as Conjecturas~\ref{conj:make-bipartite} e~\ref{conj:metadinha}.
Recentemente, o método de álgebras de flag também foi utilizado para obter resultados parciais para a Conjectura~\ref{conj:metadinha} (trocando a constante $1/50$ por $27/1024$).
Dessa forma, espera-se que os produtos desse trabalho possam ser utilizados em problemas relacionados da área.

\newcommand{\cala}{\mathcal{A}}
\newcommand{\calf}{\mathcal{F}}
\newcommand{\calk}{\mathcal{K}}
\newcommand{\calp}{\mathcal{P}}
\newcommand{\bbe}{\mathbb{E}}
\newcommand{\bbr}{\mathbb{R}}
\newcommand{\bbs}{\mathbb{S}}


\newcommand{\rfsigma}{\mathbb{R}\mathcal{F}^\sigma}
\newcommand{\fsigma}{\mathcal{F}^\sigma}
\newcommand{\asigma}{\mathcal{A}^\sigma}

\newcommand{\emptyflag}{\varnothing}
\newcommand{\aempty}{\mathcal{A}^\emptyflag}

\newcommand{\isom}{\cong}
\newcommand{\average}[1]{\left\llbracket{#1}\right\rrbracket}

\newcommand{\downward}{\downarrow\!\!}

% Tikz setup from https://arxiv.org/abs/2103.14179


\newcommand{\flagcode}[1]{\ensuremath{\vcenter{\hbox{\begin{tikzpicture}[scale=0.3]{#1}\end{tikzpicture}}}}}
\tikzset{vtx/.style={inner sep=1.2pt, outer sep=0pt, circle, fill}}
\tikzset{unlabeled_vertex/.style={inner sep=1.2pt, outer sep=0pt, circle, fill, draw=black}}
\tikzset{labeled_vertex/.style={inner sep=2.2pt, outer sep=0pt, rectangle, fill=yellow, draw=black}}
\tikzset{edge_color0/.style={color=black,line width=0.9pt}}
\tikzset{edge_color1/.style={color=red,  line width=0.9pt,opacity=0}}
\tikzset{edge_color2/.style={color=blue, line width=0.9pt,opacity=1}}

\newcommand{\edge}{ % this is the unlabeled edge
  \flagcode{
    \draw (225:0.8) coordinate(x0);
    \draw (45:0.8) coordinate(x1);
    \draw[edge_color2] (x0)--(x1);
    \draw (x0) node[unlabeled_vertex]{};
    \draw (x1) node[unlabeled_vertex]{};
  }
}

\newcommand{\labeledEdge}{ % this is edge with two label vertices
  \flagcode{
    \draw (180:0.8) coordinate(x0);
    \draw (0:0.8) coordinate(x1);
    \draw[edge_color2] (x0)--(x1);
    \draw (x0) node[labeled_vertex,label=below:$1$]{};
    \draw (x1) node[labeled_vertex,label=below:$2$]{};
  }
}

\newcommand{\kThree}{ % this is the unlabeled triangle
  \flagcode{
    \draw \foreach \x in {0,1,2}{(270+\x*360/3:0.8) coordinate(x\x)};
    \draw[edge_color2] (x0)--(x1)--(x2)--(x0);
    \draw (x0) node[unlabeled_vertex]{};
    \draw (x1) node[unlabeled_vertex]{};
    \draw (x2) node[unlabeled_vertex]{};
  }
}

\newcommand{\threePoints}{ % this is the "empty triangle"
  \flagcode{
    \draw \foreach \x in {0,1,2}{(270+\x*360/3:0.8) coordinate(x\x)};
    \draw (x0) node[unlabeled_vertex]{};
    \draw (x1) node[unlabeled_vertex]{};
    \draw (x2) node[unlabeled_vertex]{};
  }
}

\newcommand{\threePointsWithOneLabel}{ % this is the "empty triangle" with a special vertex
  \flagcode{
    \draw \foreach \x in {0,1,2}{(270+\x*360/3:0.8) coordinate(x\x)};
    \draw (x0) node[labeled_vertex]{};
    \draw (x1) node[unlabeled_vertex]{};
    \draw (x2) node[unlabeled_vertex]{};
  }
}

\newcommand{\unlabeledCherry}{ % this is three points with two edges
  \flagcode{
    \draw \foreach \x in {0,1,2}{(270+\x*360/3:0.8) coordinate(x\x)};
    \draw[edge_color2] (x1)--(x0)--(x2);
    \draw (x0) node[unlabeled_vertex]{};
    \draw (x1) node[unlabeled_vertex]{};
    \draw (x2) node[unlabeled_vertex]{};
  }
}

\newcommand{\unlabeledCherryComplement}{ % this is three points with only one edge
  \flagcode{
    \draw \foreach \x in {0,1,2}{(270+\x*360/3:0.8) coordinate(x\x)};
    \draw[edge_color2] (x1)--(x2);
    \draw (x0) node[unlabeled_vertex]{};
    \draw (x1) node[unlabeled_vertex]{};
    \draw (x2) node[unlabeled_vertex]{};
  }
}

\newcommand{\aSinglePoint}{ % this a single (labeled) point
  \flagcode{
    \draw (0,0) node[labeled_vertex]{};
  }
}

\newcommand{\edgeWithOneLabel}{ % this is the edge with one labeled vertex
  \flagcode{
    \draw (225:0.8) coordinate(x0);
    \draw (45:0.8) coordinate(x1);
    \draw[edge_color2] (x0)--(x1);
    \draw (x0) node[labeled_vertex]{};
    \draw (x1) node[unlabeled_vertex]{};
  }
}

\newcommand{\nonEdgeWithOneLabel}{ % this is the nonedge with one labeled vertex
  \flagcode{
    \draw (225:0.8) coordinate(x0);
    \draw (45:0.8) coordinate(x1);
    \draw (x0) node[labeled_vertex]{};
    \draw (x1) node[unlabeled_vertex]{};
  }
}

\newcommand{\labeledCherry}{ % this is the cherry centered at a special point
  \flagcode{
    \draw \foreach \x in {0,1,2}{(270+\x*360/3:0.8) coordinate(x\x)};
    \draw[edge_color2] (x2)--(x0)--(x1);
    \draw (x0) node[labeled_vertex]{};
    \draw (x1) node[unlabeled_vertex]{};
    \draw (x2) node[unlabeled_vertex]{};
  }
}

\newcommand{\labeledCherryComplement}{ % this is the complement of the cherry
  \flagcode{
    \draw \foreach \x in {0,1,2}{(270+\x*360/3:0.8) coordinate(x\x)};
    \draw[edge_color2] (x1)--(x2);
    \draw (x0) node[labeled_vertex]{};
    \draw (x1) node[unlabeled_vertex]{};
    \draw (x2) node[unlabeled_vertex]{};
  }
}

\newcommand{\kThreeWithOneLabel}{ % this is the triangle with a special vertex
  \flagcode{
    \draw \foreach \x in {0,1,2}{(270+\x*360/3:0.8) coordinate(x\x)};
    \draw[edge_color2] (x2)--(x0)--(x1)--(x2);
    \draw (x0) node[labeled_vertex]{};
    \draw (x1) node[unlabeled_vertex]{};
    \draw (x2) node[unlabeled_vertex]{};
  }
}

\newcommand{\asymmetricLabeledCherry}{ % this is like a labeled cherry but the special vertex is a leaf
  \flagcode{
    \draw \foreach \x in {0,1,2}{(270+\x*360/3:0.8) coordinate(x\x)};
    \draw[edge_color2] (x0)--(x1)--(x2);
    \draw (x0) node[labeled_vertex]{};
    \draw (x1) node[unlabeled_vertex]{};
    \draw (x2) node[unlabeled_vertex]{};
  }
}

\newcommand{\asymmetricLabeledCherryComplement}{ % this is the complement of the asymmetric labeled cherry
  \flagcode{
    \draw \foreach \x in {0,1,2}{(270+\x*360/3:0.8) coordinate(x\x)};
    \draw[edge_color2] (x0)--(x1);
    \draw (x0) node[labeled_vertex]{};
    \draw (x1) node[unlabeled_vertex]{};
    \draw (x2) node[unlabeled_vertex]{};
  }
}

\newcommand{\cherryWithLabeledEdge}{ % this is a labeled edge and an extra edge joined to 1
  \flagcode{
    \draw \foreach \x in {0,1,2}{(-270-\x*360/3:0.8) coordinate(x\x)};
    \draw[edge_color2] (x1)--(x2)--(x0);
    \draw (x0) node[unlabeled_vertex]{};
    \draw (x1) node[labeled_vertex,label=below:$2$]{};
    \draw (x2) node[labeled_vertex,label=below:$1$]{};
  }
}

\newcommand{\anotherCherryWithLabeledEdge}{ % this is a labeled edge and an extra edge joined to 2
  \flagcode{
    \draw \foreach \x in {0,1,2}{(-270-\x*360/3:0.8) coordinate(x\x)};
    \draw[edge_color2] (x2)--(x1)--(x0);
    \draw (x0) node[unlabeled_vertex]{};
    \draw (x1) node[labeled_vertex,label=below:$2$]{};
    \draw (x2) node[labeled_vertex,label=below:$1$]{};
  }
}

\newcommand{\fourVerticesOne}{ % this is a labeled edge and an extra edge joined to 2
  \flagcode{
    \draw \foreach \x in {0,1,2,3}{(45+\x*360/4:1.5) coordinate(x\x)};
    \draw[edge_color2] (x0)--(x1);
    \draw[edge_color2] (x2)--(x3);
    \draw[edge_color2] (x1)--(x2);
    \draw (x0) node[unlabeled_vertex]{};
    \draw (x1) node[unlabeled_vertex]{};
    \draw (x2) node[labeled_vertex,label=below:$1$]{};
    \draw (x3) node[labeled_vertex,label=below:$2$]{};
  }
}


\newcommand{\fourVerticesTwo}{ % this is a labeled edge and an extra edge joined to 2
  \flagcode{
    \draw \foreach \x in {0,1,2,3}{(45+\x*360/4:1.5) coordinate(x\x)};
    \draw[edge_color2] (x0)--(x1);
    \draw[edge_color2] (x2)--(x3);
    \draw[edge_color2] (x0)--(x3);
    \draw (x0) node[unlabeled_vertex]{};
    \draw (x1) node[unlabeled_vertex]{};
    \draw (x2) node[labeled_vertex,label=below:$1$]{};
    \draw (x3) node[labeled_vertex,label=below:$2$]{};
  }
}

\newcommand{\fourVerticesThree}{ % this is a labeled edge and an extra edge joined to 2
  \flagcode{
    \draw \foreach \x in {0,1,2,3}{(45+\x*360/4:1.5) coordinate(x\x)};
    \draw[edge_color2] (x0)--(x1);
    \draw[edge_color2] (x2)--(x3);
    \draw (x0) node[unlabeled_vertex]{};
    \draw (x1) node[unlabeled_vertex]{};
    \draw (x2) node[labeled_vertex,label=below:$1$]{};
    \draw (x3) node[labeled_vertex,label=below:$2$]{};
  }
}